% Options for packages loaded elsewhere
\PassOptionsToPackage{unicode}{hyperref}
\PassOptionsToPackage{hyphens}{url}
\PassOptionsToPackage{dvipsnames,svgnames,x11names}{xcolor}
%
\documentclass[
  letterpaper,
  DIV=11,
  numbers=noendperiod]{scrreprt}

\usepackage{amsmath,amssymb}
\usepackage{iftex}
\ifPDFTeX
  \usepackage[T1]{fontenc}
  \usepackage[utf8]{inputenc}
  \usepackage{textcomp} % provide euro and other symbols
\else % if luatex or xetex
  \usepackage{unicode-math}
  \defaultfontfeatures{Scale=MatchLowercase}
  \defaultfontfeatures[\rmfamily]{Ligatures=TeX,Scale=1}
\fi
\usepackage{lmodern}
\ifPDFTeX\else  
    % xetex/luatex font selection
\fi
% Use upquote if available, for straight quotes in verbatim environments
\IfFileExists{upquote.sty}{\usepackage{upquote}}{}
\IfFileExists{microtype.sty}{% use microtype if available
  \usepackage[]{microtype}
  \UseMicrotypeSet[protrusion]{basicmath} % disable protrusion for tt fonts
}{}
\makeatletter
\@ifundefined{KOMAClassName}{% if non-KOMA class
  \IfFileExists{parskip.sty}{%
    \usepackage{parskip}
  }{% else
    \setlength{\parindent}{0pt}
    \setlength{\parskip}{6pt plus 2pt minus 1pt}}
}{% if KOMA class
  \KOMAoptions{parskip=half}}
\makeatother
\usepackage{xcolor}
\setlength{\emergencystretch}{3em} % prevent overfull lines
\setcounter{secnumdepth}{5}
% Make \paragraph and \subparagraph free-standing
\ifx\paragraph\undefined\else
  \let\oldparagraph\paragraph
  \renewcommand{\paragraph}[1]{\oldparagraph{#1}\mbox{}}
\fi
\ifx\subparagraph\undefined\else
  \let\oldsubparagraph\subparagraph
  \renewcommand{\subparagraph}[1]{\oldsubparagraph{#1}\mbox{}}
\fi


\providecommand{\tightlist}{%
  \setlength{\itemsep}{0pt}\setlength{\parskip}{0pt}}\usepackage{longtable,booktabs,array}
\usepackage{calc} % for calculating minipage widths
% Correct order of tables after \paragraph or \subparagraph
\usepackage{etoolbox}
\makeatletter
\patchcmd\longtable{\par}{\if@noskipsec\mbox{}\fi\par}{}{}
\makeatother
% Allow footnotes in longtable head/foot
\IfFileExists{footnotehyper.sty}{\usepackage{footnotehyper}}{\usepackage{footnote}}
\makesavenoteenv{longtable}
\usepackage{graphicx}
\makeatletter
\def\maxwidth{\ifdim\Gin@nat@width>\linewidth\linewidth\else\Gin@nat@width\fi}
\def\maxheight{\ifdim\Gin@nat@height>\textheight\textheight\else\Gin@nat@height\fi}
\makeatother
% Scale images if necessary, so that they will not overflow the page
% margins by default, and it is still possible to overwrite the defaults
% using explicit options in \includegraphics[width, height, ...]{}
\setkeys{Gin}{width=\maxwidth,height=\maxheight,keepaspectratio}
% Set default figure placement to htbp
\makeatletter
\def\fps@figure{htbp}
\makeatother
\newlength{\cslhangindent}
\setlength{\cslhangindent}{1.5em}
\newlength{\csllabelwidth}
\setlength{\csllabelwidth}{3em}
\newlength{\cslentryspacingunit} % times entry-spacing
\setlength{\cslentryspacingunit}{\parskip}
\newenvironment{CSLReferences}[2] % #1 hanging-ident, #2 entry spacing
 {% don't indent paragraphs
  \setlength{\parindent}{0pt}
  % turn on hanging indent if param 1 is 1
  \ifodd #1
  \let\oldpar\par
  \def\par{\hangindent=\cslhangindent\oldpar}
  \fi
  % set entry spacing
  \setlength{\parskip}{#2\cslentryspacingunit}
 }%
 {}
\usepackage{calc}
\newcommand{\CSLBlock}[1]{#1\hfill\break}
\newcommand{\CSLLeftMargin}[1]{\parbox[t]{\csllabelwidth}{#1}}
\newcommand{\CSLRightInline}[1]{\parbox[t]{\linewidth - \csllabelwidth}{#1}\break}
\newcommand{\CSLIndent}[1]{\hspace{\cslhangindent}#1}

\KOMAoption{captions}{tableheading}
\makeatletter
\makeatother
\makeatletter
\@ifpackageloaded{bookmark}{}{\usepackage{bookmark}}
\makeatother
\makeatletter
\@ifpackageloaded{caption}{}{\usepackage{caption}}
\AtBeginDocument{%
\ifdefined\contentsname
  \renewcommand*\contentsname{Table of contents}
\else
  \newcommand\contentsname{Table of contents}
\fi
\ifdefined\listfigurename
  \renewcommand*\listfigurename{List of Figures}
\else
  \newcommand\listfigurename{List of Figures}
\fi
\ifdefined\listtablename
  \renewcommand*\listtablename{List of Tables}
\else
  \newcommand\listtablename{List of Tables}
\fi
\ifdefined\figurename
  \renewcommand*\figurename{Figure}
\else
  \newcommand\figurename{Figure}
\fi
\ifdefined\tablename
  \renewcommand*\tablename{Table}
\else
  \newcommand\tablename{Table}
\fi
}
\@ifpackageloaded{float}{}{\usepackage{float}}
\floatstyle{ruled}
\@ifundefined{c@chapter}{\newfloat{codelisting}{h}{lop}}{\newfloat{codelisting}{h}{lop}[chapter]}
\floatname{codelisting}{Listing}
\newcommand*\listoflistings{\listof{codelisting}{List of Listings}}
\makeatother
\makeatletter
\@ifpackageloaded{caption}{}{\usepackage{caption}}
\@ifpackageloaded{subcaption}{}{\usepackage{subcaption}}
\makeatother
\makeatletter
\@ifpackageloaded{tcolorbox}{}{\usepackage[skins,breakable]{tcolorbox}}
\makeatother
\makeatletter
\@ifundefined{shadecolor}{\definecolor{shadecolor}{rgb}{.97, .97, .97}}
\makeatother
\makeatletter
\makeatother
\makeatletter
\makeatother
\ifLuaTeX
  \usepackage{selnolig}  % disable illegal ligatures
\fi
\IfFileExists{bookmark.sty}{\usepackage{bookmark}}{\usepackage{hyperref}}
\IfFileExists{xurl.sty}{\usepackage{xurl}}{} % add URL line breaks if available
\urlstyle{same} % disable monospaced font for URLs
\hypersetup{
  pdftitle={Jurimetria das Desiguladades},
  pdfauthor={Fabiana Severi; Alexandre Nicolella, José de Jesus Filho},
  colorlinks=true,
  linkcolor={blue},
  filecolor={Maroon},
  citecolor={Blue},
  urlcolor={Blue},
  pdfcreator={LaTeX via pandoc}}

\title{Jurimetria das Desiguladades}
\author{Fabiana Severi; Alexandre Nicolella, José de Jesus Filho}
\date{2024-06-04}

\begin{document}
\maketitle
\ifdefined\Shaded\renewenvironment{Shaded}{\begin{tcolorbox}[borderline west={3pt}{0pt}{shadecolor}, boxrule=0pt, interior hidden, breakable, sharp corners, enhanced, frame hidden]}{\end{tcolorbox}}\fi

\renewcommand*\contentsname{Table of contents}
{
\hypersetup{linkcolor=}
\setcounter{tocdepth}{2}
\tableofcontents
}
\bookmarksetup{startatroot}

\hypertarget{prefuxe1cio}{%
\chapter{Prefácio}\label{prefuxe1cio}}

\hypertarget{nossa-motivauxe7uxe3o}{%
\section{\texorpdfstring{\textbf{Nossa
Motivação:}}{Nossa Motivação:}}\label{nossa-motivauxe7uxe3o}}

Esse curso foi criado pensando em profissionais com formação em direito,
estatística, tecnologia da informação e demais profissionais que atuam
tanto no judiciário quanto em tribunais administrativos. A ideia é
colocar juntas diversas visões sobre um mesmo problema para buscarmos
soluções mais criativas e eficicientes para as desigualdades nacionais.

\hypertarget{nosso-objetivo}{%
\section{\texorpdfstring{\textbf{Nosso
Objetivo}:}{Nosso Objetivo:}}\label{nosso-objetivo}}

É oferecer aos participantes uma abordagem probabilística do direito
aplicado, tendo como foco os casos práticos e discutindo questões
relativas ao direito e desigualdades. Especificamente, objetiva-se
ensinar um conjunto de métodos quantitativos e oferecer ferramentas para
coleta, transformação e análise de dados jurídicos disponibilizados nas
páginas dos tribunais de justiça.

\hypertarget{nosso-problema}{%
\section{\texorpdfstring{\textbf{Nosso
Problema}:}{Nosso Problema:}}\label{nosso-problema}}

Teremos como tema de fundo nesse curso os processos de homicídio
feminino. Especificamente queremos compreender qual o perfil e os
determinantes do feminicidio no Estado de São Paulo.

\bookmarksetup{startatroot}

\hypertarget{introduuxe7uxe3o-ao-feminicuxeddio}{%
\chapter{Introdução ao
Feminicídio}\label{introduuxe7uxe3o-ao-feminicuxeddio}}

Nessa seção pensei em falarmos um pouco no no caso referências
etc\ldots.

\bookmarksetup{startatroot}

\hypertarget{sumuxe1rio-do-curso}{%
\chapter{Sumário do Curso}\label{sumuxe1rio-do-curso}}

\hypertarget{resumo}{%
\section{\texorpdfstring{\textbf{Resumo}:}{Resumo:}}\label{resumo}}

A pesquisa no direito e a ciência de dados: introdução, relevância e
experiências; Coleta, transformação e estruturação de dados processuais
com R; Estatística descritiva; Aprendizado estatístico (regressão
linear, regressão logística); Aprendizado estatístico (\emph{machine
learning}); Interpretação dos resultados e elaboração de relatório.

\hypertarget{plano-detalhado-das-aulas}{%
\section{\texorpdfstring{\textbf{Plano detalhado das
aulas:}}{Plano detalhado das aulas:}}\label{plano-detalhado-das-aulas}}

\begin{enumerate}
\def\labelenumi{\arabic{enumi}.}
\item
  A pesquisa no direito e a ciência de dados: introdução, relevância e
  experiências - 25/04
\item
  Coleta, transformação e estruturação de dados processuais com R -
  02/05
\item
  Coleta, transformação e estruturação de dados processuais com R -
  09/05
\item
  Análise descritiva e visualização de dados processuais - 16/05
\item
  Modelos lineares aplicados a pesquisa em direito -- regressão linear -
  23/05
\item
  Modelos para diferentes tipos de estrutura de dados e variáveis. -
  06/06
\item
  Aprendizado estatístico (\emph{machine learning}) - 13/06
\item
  Aprendizado estatístico (\emph{machine learning}) - 20/06
\item
  Interpretação dos resultados e elaboração do relatório - 27/06
\end{enumerate}

\bookmarksetup{startatroot}

\hypertarget{a-pesquisa-no-direito-e-a-ciuxeancia-de-dados}{%
\chapter{A pesquisa no direito e a ciência de
dados}\label{a-pesquisa-no-direito-e-a-ciuxeancia-de-dados}}

\bookmarksetup{startatroot}

\hypertarget{section}{%
\chapter{}\label{section}}

\bookmarksetup{startatroot}

\hypertarget{references}{%
\chapter*{References}\label{references}}
\addcontentsline{toc}{chapter}{References}

\markboth{References}{References}

\hypertarget{refs}{}
\begin{CSLReferences}{0}{0}
\end{CSLReferences}



\end{document}
